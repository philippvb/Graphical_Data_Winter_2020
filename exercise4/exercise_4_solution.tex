\newcommand{\NUMBER}{4}
\newcommand{\EXERCISES}{3}
% \newcommand{\DEADLINE}{16.11.20}
\newcommand{\COURSE}{Graphical Data}
\newcommand{\STUDENTA}{Philipp von Bachmann, \\Mat.-Nr.: 4116220}
\newcommand{\STUDENTB}{David Ott, \\Mat.-Nr.: 4185646}
\documentclass[a4paper]{scrartcl}
\usepackage[utf8]{inputenc}
\usepackage[english]{babel}
\usepackage{amsmath, enumerate, amssymb, multirow, fancyhdr, color, graphicx, lastpage, listings, tikz, pdflscape, subfigure, float, polynom, hyperref, tabularx, forloop, geometry, listings, fancybox, tikz, forest, tabstackengine, cancel, hyperref}
\input kvmacros
\geometry{a4paper,left=3cm, right=3cm, top=3cm, bottom=3cm}
\pagestyle {fancy}
\fancyhead[C]{\COURSE}
\fancyhead[R]{\today}
\fancyfoot[L]{}
\fancyfoot[C]{}
\fancyfoot[R]{Page \thepage /\pageref*{LastPage}}
\def\header#1#2{
  \begin{center}
    {\Large Assignment #1}\\
    %{(Due by: #2)}
  \end{center}
}

\newcounter{punktelistectr}
\newcounter{punkte}
\newcommand{\punkteliste}[2]{%
  \setcounter{punkte}{#2}%
  \addtocounter{punkte}{-#1}%
  \stepcounter{punkte}%<-- also punkte = m-n+1 = Anzahl Spalten[1]
  \begin{center}%
  \begin{tabularx}{\linewidth}[]{@{}*{\thepunkte}{>{\centering\arraybackslash} X|}@{}>{\centering\arraybackslash}X}
      \forloop{punktelistectr}{#1}{\value{punktelistectr} < #2 } %
      {%
        \thepunktelistectr &
      }
      #2 &  $\Sigma$ \\
      \hline
      \forloop{punktelistectr}{#1}{\value{punktelistectr} < #2 } %
      {%
        &
      } &\\
      \forloop{punktelistectr}{#1}{\value{punktelistectr} < #2 } %
      {%
        &
      } &\\
    \end{tabularx}
  \end{center}
}
\begin{document}

\begin{tabularx}{\linewidth}{m{0.3 \linewidth}X}
  \begin{minipage}{\linewidth}
    \STUDENTA\\
    \STUDENTB
  \end{minipage} & \begin{minipage}{\linewidth}
    \punkteliste{1}{\EXERCISES}
  \end{minipage}\\
\end{tabularx}
\header{Nr. \NUMBER}{\DEADLINE}


\section*{4.1 Radiometry}
Let $w'$ be the direction from x to d
\begin{align*}
  E
  &=\int_\omega L(x,w)cos(\theta )d \omega\\
  &=L(x,w')cos(\theta )\\
  &=\frac{d^2 \Phi cos(\theta )}{dA cos(\theta )d\omega }\\
  \text{as a point is perpendicular to another cos =1}\\
  &=\frac{d^2 \Phi cos(\theta )}{dA 1 d\omega }\\
  \text{Assume $dA =1$, then }dw = 4\pi r^2 \\
  &=\frac{d^2 \Phi cos(\theta )}{4\pi r^2}\\
  \text{The radius is given by }r=\sqrt{r^2+d^2}\\
  &=\frac{d^2 \Phi cos(\theta )}{4\pi r^2+d^2 }\\
\end{align*}

\subsection*{a)}
$\omega = (\theta, \phi)$
\begin{align*}
  E &= \int_{0}^{2\pi} \int_0^{\frac{\pi}{2}} L(x, \omega) \cos \theta \sin \theta d \theta d \phi \\
  &= \int_{0}^{2\pi} \int_0^{\frac{\pi}{2}} \frac{d^2 \Phi}{dA \cos(\theta) d\omega} \cos \theta \sin \theta d \theta d \phi \\
  &= \int_{0}^{2\pi} \int_0^{\frac{\pi}{2}} \frac{d^2 \Phi}{\sin(\theta) d \theta d \phi \cdot r^2 \cos(\theta) d\omega} \cos \theta \sin \theta d \theta d \phi \\
  &= \int_{0}^{2\pi} \int_0^{\frac{\pi}{2}} \frac{d^2 \Phi}{r^2 d \omega} \\
  &= \int_{0}^{2\pi} \int_0^{\frac{\pi}{2}} \frac{d^2 \Phi}{r^2 d \theta d \phi} \\
  &= \\
  &= \frac{\Phi_S \cos(\theta)}{4 \pi(r^2 + d^2)}
\end{align*}

\section*{4.2 Analytical solution of the rendering equation in 2D}
  \subsection*{a)}
    \begin{align*}
      L(x,\omega_o)
      &=L_e(x,\omega_o) + \int_{0}^\pi f_r(\omega_i, x, \omega_o) L(x,\omega_i)cos(\theta_i)d\omega_i\\
      &=0 + \int_{0}^\pi \frac{1}{2} L(x,\omega_i)cos(\theta_i)d\omega_i\\
      &=\frac{1}{2}\int_{0}^\pi L(x,\omega_i)cos(\theta_i)d\omega_i\\
      &=\frac{1}{2}\int_{0}^\pi L(x,\omega_i)cos(\theta_i)d\omega_i\\
      &=\frac{1}{2}(\int_{0}^{tan^{-1}(\frac{1}{p+1 })} L(x,\omega_i)cos(\theta_i)d\omega_i + \int_{tan^{-1}(\frac{1}{p+1 })}^{\pi-tan^{-1}(\frac{1}{1-p})} L(x,\omega_i)cos(\theta_i)d\omega_i + \int_{\pi-tan^{-1}(\frac{1}{1-p})}^{\pi} L(x,\omega_i)cos(\theta_i)d\omega_i)\\
      &\text{The first and the third integral refer to incident angles that point past the light source, therefore they are 0}\\
      &=\frac{1}{2} \int_{tan^{-1}(\frac{1}{p+1 })}^{\pi-tan^{-1}(\frac{1}{1-p})} 1\cdot cos(\theta_i)d\omega_i\\
      &=\frac{1}{2} \int_{tan^{-1}(\frac{1}{p+1 })}^{\pi-tan^{-1}(\frac{1}{1-p})} cos(\lvert \omega_i - \frac{\pi}{2} \rvert)d\omega_i \\
      &=\frac{1}{2} \left( \int_{tan^{-1}(\frac{1}{p+1 })}^{\frac{\pi}{2}} cos(\frac{\pi}{2} - \omega_i )d\omega_i +  \int_{\frac{\pi}{2}}^{\pi-tan^{-1}(\frac{1}{1-p})}cos(\omega_i - \frac{\pi}{2} )d\omega_i\right)\\
    \end{align*}



\subsection*{b)}

\begin{align*}
  L(x, \omega_o) &= L_e(x, \omega_o) + \int_{y \in S} f_r(\omega_i, x, \omega_o) \cdot L(y, -\omega_i(x,y)) \cdot \frac{\cos \phi_i \cos \phi_y}{\lvert x - y \rvert} \cdot d y_x \\
  &= L_e(x, \omega_o) + \int_{-1}^{1} f_r(\omega_i, x, \omega_o) \cdot L(y, -\omega_i(x,y)) \cdot \frac{\cos \phi_i \cos \phi_y}{\lvert x - y \rvert} \cdot d y_x \\
  & \text{If we are not in a light source this is: } \\
  &= \int_{-1}^{1} f_r(\omega_i, x, \omega_o) \cdot L(y, -\omega_i(x,y)) \cdot \frac{\cos \phi_i \cos \phi_y}{\lvert x - y \rvert} \cdot d y_x \\
  &= \int_{-1}^{1} f_r(\omega_i, x, \omega_o) \cdot L(y, -\omega_i(x,y)) \cdot \frac{\cos \phi_i \cos \phi_y}{\sqrt{(x_x - y_x)^2 + (x_y - y_y)^2}} \cdot d y_x \\
  &= \int_{-1}^{1} f_r(\omega_i, x, \omega_o) \cdot 1 \cdot \frac{\cos \phi_i \cos \phi_y}{\sqrt{(x_x - y_x)^2 + (x_y - y_y)^2}} \cdot d y_x \\
  &= 
\end{align*}


\section*{4.3 Simple Path Tracer}


\end{document}
