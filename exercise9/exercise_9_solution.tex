\newcommand{\NUMBER}{9}
\newcommand{\EXERCISES}{4}
% \newcommand{\DEADLINE}{16.11.20}
\newcommand{\COURSE}{Graphical Data}
\newcommand{\STUDENTA}{Philipp von Bachmann, \\Mat.-Nr.: 4116220}
\newcommand{\STUDENTB}{David Ott, \\Mat.-Nr.: 4185646}
\documentclass[a4paper]{scrartcl}
\usepackage[utf8]{inputenc}
\usepackage[english]{babel}
\usepackage{amsmath, enumerate, amssymb, multirow, fancyhdr, color, graphicx, lastpage, listings, tikz, pdflscape, subfigure, float, polynom, hyperref, tabularx, forloop, geometry, listings, fancybox, tikz, forest, tabstackengine, cancel, hyperref}
\input kvmacros
\geometry{a4paper,left=3cm, right=3cm, top=3cm, bottom=3cm}
\pagestyle {fancy}
\fancyhead[C]{\COURSE}
\fancyhead[R]{\today}
\fancyfoot[L]{}
\fancyfoot[C]{}
\fancyfoot[R]{Page \thepage /\pageref*{LastPage}}
\def\header#1#2{
  \begin{center}
    {\Large Assignment #1}\\
    %{(Due by: #2)}
  \end{center}
}

\newcounter{punktelistectr}
\newcounter{punkte}
\newcommand{\punkteliste}[2]{%
  \setcounter{punkte}{#2}%
  \addtocounter{punkte}{-#1}%
  \stepcounter{punkte}%<-- also punkte = m-n+1 = Anzahl Spalten[1]
  \begin{center}%
  \begin{tabularx}{\linewidth}[]{@{}*{\thepunkte}{>{\centering\arraybackslash} X|}@{}>{\centering\arraybackslash}X}
      \forloop{punktelistectr}{#1}{\value{punktelistectr} < #2 } %
      {%
        \thepunktelistectr &
      }
      #2 &  $\Sigma$ \\
      \hline
      \forloop{punktelistectr}{#1}{\value{punktelistectr} < #2 } %
      {%
        &
      } &\\
      \forloop{punktelistectr}{#1}{\value{punktelistectr} < #2 } %
      {%
        &
      } &\\
    \end{tabularx}
  \end{center}
}
\begin{document}

\begin{tabularx}{\linewidth}{m{0.3 \linewidth}X}
  \begin{minipage}{\linewidth}
    \STUDENTA\\
    \STUDENTB
  \end{minipage} & \begin{minipage}{\linewidth}
    \punkteliste{1}{\EXERCISES}
  \end{minipage}\\
\end{tabularx}
\header{Nr. \NUMBER}{\DEADLINE}

\section*{6.1 Duality of Multiplication and Convolution}


\section*{6.2 Fourier Transformation}
  \begin{align*}
    B(k)
    &=\int_{-\infty}^\infty b(x) e^{-2\pi i kx} dx\\
    &=\int_{-1}^1 e^{-2\pi i kx} dx\\
    &=\int_{-1}^1 cos(-2\pi kx) + i \cdot sin(-2\pi kx) dx\\
    &=[\frac{1}{-2\pi k } sin(-2\pi kx) + i \cdot \frac{1}{-2 \pi k}(- cos(-2\pi kx))]_{-1}^1 \\
    &=[\frac{1}{-2\pi k } (sin(-2\pi kx) - i \cdot cos(-2\pi kx))]_{-1}^1 \\
    &=\frac{1}{-2\pi k} \Big( \big( sin(-2\pi k) - i \cdot cos(-2\pi k)\big) -\big( sin(2\pi k) - i \cdot cos(2\pi k) \big) \Big) \\
    &=\frac{1}{-2\pi k} \Big( sin(-2\pi k) - i \cdot cos(-2\pi k) - sin(2\pi k) + i \cdot cos(2\pi k) \Big) \\
    &=\frac{1}{-2\pi k} \Big( sin(-2\pi k) - sin(2\pi k)\Big) \\
    &=\frac{1}{-2\pi k} \Big( 2 sin(\frac{-2 \pi k - 2 \pi k }{2})\Big) \\
    &=\frac{1}{-2\pi k} \Big( 2 sin(-2\pi k)\Big) \\
    &=\frac{sin(-2\pi k)}{-\pi k}\\
  \end{align*}
  

\end{document}
