\newcommand{\NUMBER}{3}
\newcommand{\EXERCISES}{4}
% \newcommand{\DEADLINE}{16.11.20}
\newcommand{\COURSE}{Graphical Data}
\newcommand{\STUDENTA}{Philipp von Bachmann, \\Mat.-Nr.: 4116220}
\newcommand{\STUDENTB}{David Ott, \\Mat.-Nr.: 4185646}
\documentclass[a4paper]{scrartcl}
\usepackage[utf8]{inputenc}
\usepackage[english]{babel}
\usepackage{amsmath, enumerate, amssymb, multirow, fancyhdr, color, graphicx, lastpage, listings, tikz, pdflscape, subfigure, float, polynom, hyperref, tabularx, forloop, geometry, listings, fancybox, tikz, forest, tabstackengine, cancel, hyperref}
\input kvmacros
\geometry{a4paper,left=3cm, right=3cm, top=3cm, bottom=3cm}
\pagestyle {fancy}
\fancyhead[C]{\COURSE}
\fancyhead[R]{\today}
\fancyfoot[L]{}
\fancyfoot[C]{}
\fancyfoot[R]{Page \thepage /\pageref*{LastPage}}
\def\header#1#2{
  \begin{center}
    {\Large Assignment #1}\\
    %{(Due by: #2)}
  \end{center}
}

\newcounter{punktelistectr}
\newcounter{punkte}
\newcommand{\punkteliste}[2]{%
  \setcounter{punkte}{#2}%
  \addtocounter{punkte}{-#1}%
  \stepcounter{punkte}%<-- also punkte = m-n+1 = Anzahl Spalten[1]
  \begin{center}%
  \begin{tabularx}{\linewidth}[]{@{}*{\thepunkte}{>{\centering\arraybackslash} X|}@{}>{\centering\arraybackslash}X}
      \forloop{punktelistectr}{#1}{\value{punktelistectr} < #2 } %
      {%
        \thepunktelistectr &
      }
      #2 &  $\Sigma$ \\
      \hline
      \forloop{punktelistectr}{#1}{\value{punktelistectr} < #2 } %
      {%
        &
      } &\\
      \forloop{punktelistectr}{#1}{\value{punktelistectr} < #2 } %
      {%
        &
      } &\\
    \end{tabularx}
  \end{center}
}
\begin{document}

\begin{tabularx}{\linewidth}{m{0.3 \linewidth}X}
  \begin{minipage}{\linewidth}
    \STUDENTA\\
    \STUDENTB
  \end{minipage} & \begin{minipage}{\linewidth}
    \punkteliste{1}{\EXERCISES}
  \end{minipage}\\
\end{tabularx}
\header{Nr. \NUMBER}{\DEADLINE}


\section*{3.1 Reflection Rays}

In general, the direction of a ray reflected at a normal is 
$$ d_{ref} = 2 (n \cdot d) n-d$$

We have to keep in mind that the ray that is reflected should point in the
correct (therefore reversed) direction and that it starts at the new time
$t_{hit}$. therefore the complete formula for the new ray is:



\begin{align*}
  r_{ref}(t) = o + t_{hit}d - (t-t_{hit}) (2(n \cdot d) n-d)
\end{align*}

\section*{3.2 Front/Back-side of a triangle}
The scalar product is $< 0$ for angles between two vectors greater than
$90^\circ$ and $>0$ for angles between two vectors smaller than $90^\circ$.

So the ray hits the triangle from the front, if $n \cdot d > 0$ and it hits it
from the back if $n \cdot d < 0$. 



\end{document}
