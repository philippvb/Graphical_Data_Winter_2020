\newcommand{\NUMBER}{1}
\newcommand{\EXERCISES}{3}
\newcommand{\DEADLINE}{16.11.20}
\newcommand{\COURSE}{Graphical Data}
\newcommand{\STUDENTA}{Philipp von Bachmann}
\newcommand{\STUDENTB}{David Ott}
\documentclass[a4paper]{scrartcl}
\usepackage[utf8]{inputenc}
\usepackage[english]{babel}
\usepackage{amsmath, enumerate, amssymb, multirow, fancyhdr, color, graphicx, lastpage, listings, tikz, pdflscape, subfigure, float, polynom, hyperref, tabularx, forloop, geometry, listings, fancybox, tikz, forest, tabstackengine, cancel, hyperref}
\input kvmacros
\geometry{a4paper,left=3cm, right=3cm, top=3cm, bottom=3cm}
\pagestyle {fancy}
\fancyhead[C]{\COURSE}
\fancyhead[R]{\today}
\fancyfoot[L]{}
\fancyfoot[C]{}
\fancyfoot[R]{Page \thepage /\pageref*{LastPage}}
\def\header#1#2{
  \begin{center}
    {\Large Assignment #1}\\
    %{(Due by: #2)}
  \end{center}
}

\newcounter{punktelistectr}
\newcounter{punkte}
\newcommand{\punkteliste}[2]{%
  \setcounter{punkte}{#2}%
  \addtocounter{punkte}{-#1}%
  \stepcounter{punkte}%<-- also punkte = m-n+1 = Anzahl Spalten[1]
  \begin{center}%
  \begin{tabularx}{\linewidth}[]{@{}*{\thepunkte}{>{\centering\arraybackslash} X|}@{}>{\centering\arraybackslash}X}
      \forloop{punktelistectr}{#1}{\value{punktelistectr} < #2 } %
      {%
        \thepunktelistectr &
      }
      #2 &  $\Sigma$ \\
      \hline
      \forloop{punktelistectr}{#1}{\value{punktelistectr} < #2 } %
      {%
        &
      } &\\
      \forloop{punktelistectr}{#1}{\value{punktelistectr} < #2 } %
      {%
        &
      } &\\
    \end{tabularx}
  \end{center}
}
\begin{document}

\begin{tabularx}{\linewidth}{m{0.3 \linewidth}X}
  \begin{minipage}{\linewidth}
    \STUDENTA\\
    \STUDENTB
  \end{minipage} & \begin{minipage}{\linewidth}
    \punkteliste{1}{\EXERCISES}
  \end{minipage}\\
\end{tabularx}
\header{Nr. \NUMBER}{\DEADLINE}


\section*{1.1 Primary Ray-Generation for a Perspective Camera Model}
To get to the center of the image plane from the origin, we have to move in
direction of the viewing direction $d$: $f\cdot d$. To move in $y$ direction
in the image plane, the vector is already given as the up-vector $u$. As the
vector in $x$ direction is perpendicular to both $d$ and $u$, we can construct
it by $v=d\times u$. Because $u$ and $d$ are normalized, $v$ is also normalized.
To computed how far we have to move in $x$ direction from the center of our
image plane for a given coordinate $x$, we compute $x_{diff}=x-\frac{resX}{2}$.
Similar for $y$ direction. In combination, we end up with
$$d_r = f \cdot d + (x - \frac{resX}{2}) \cdot (d \times u) + (y - \frac{resY}{2}) \cdot u$$
With this definition, the origin of the image plane (pixel coordinates 0,0) is at the lower left.

\section*{1.2 Ray-Surface Intersection}
    \subsection*{a)}
        As the plane is defined by $(p-n)\cdot a=0$, in order to check if $r(t)$
        is in the plane, we have to find a $t$ such that $(r(t)-n)\cdot a=0$.
        \begin{align*}
            &(r(t)-a)\cdot n=0\\
            &((o+t\cdot d)-a)\cdot n=0\\
            &(o+t\cdot d) \cdot n -a\cdot n=0\\
            &(o+t\cdot d) \cdot n =a\cdot n\\
            &o\cdot n+t\cdot d\cdot n =a\cdot n\\
            &t\cdot d\cdot n =a\cdot n-o\cdot n\\
            &t\cdot d\cdot n =a\cdot n-o\cdot n\\
            &t=\frac{a\cdot n-o\cdot n}{d\cdot n}\\
        \end{align*}
        This fraction exists if $d\cdot n \neq 0$, or in other words the vector
        $d$ of the ray is not parallel to the plane. 
    \subsection*{b)}
        For the ray to be possibly in the bounding box for the x-,y-, or z-Direction it needs to hold:
            
        \begin{align*}{}
          min_i \leq r(t)_i \leq max_i \\
          min_i \leq o_i + t d_i \leq max_i \\
        \end{align*}
        for $d_i > 0$ this is:
        \begin{align*}
          \frac{min_i - o_i}{d_i} \leq t \leq \frac{max_i - o_i}{d_i}          
        \end{align*} 
        for $d_i < 0$ this is:
        \begin{align*}
          \frac{max_i - o_i}{d_i} \leq t \leq \frac{min_i - o_i}{d_i}          
        \end{align*} 

        If $d_i = 0$ we check if $min_i < o_i < max_i$. If this is the case we set 
        ${t_{low}}_i = -\infty$ and ${t_{high}}_i = \infty$. Else there are no intersections with the bounding box
        
        In this way we have defined ranges for $t$ for each of the dimensions. 
        To find the total range of $t$ we have to take the maximum of the lower ends of the 
        ranges and the minimum of the upper ends of the ranges.

        We calculate $t_{low} = \frac{min - o}{d}$ and $t_{high} = \frac{max - o}{d}$
        for each dimension (x,y,z) we swap ${t_{low}}_i$ and ${t_{high}}_i$ if $d_i < 0$ to get $t'_{low}$ and $t'_{high}$.
        
        The ray is in the bounding box for $\max(t'_{low}) \leq t \leq \min(t'_{high})$. If $\max(t'_{low}) > \min(t'_{high})$, then
        there are no intersections of the ray with the bounding box. The first intersection of the ray with the bounding box is for 
        $\max(t'_{low})$


    \subsection*{c)}
        There are several ways to do this, here we show an algebraic check:\\
        $r(t)$ can only pass through the triangle at the point where it crosses 
        the plane that is extending the triangle.
        We get the normal of the plane
        by computing $n=(p_2 - p_1)\times (p_3 - p_1)$.\\
        If we set $a=p_3$, we can compute the intersection point $q$ of $r(t)$
        and the plane according to 1.2 a). 
        
        In order to check if $q$ is in the triangle we need to check if this point is on
        that side of each of the lines that enclose the triangle that is the same side as the 
        triangle. For example the cross product of $p_2 - p_1$ with $q - p_1$ should point in 
        the same direction as the normal of the plane. 

        For all lines this is done by checking:
        \begin{align*}
          ((p_2 - p_1) \times (q - p_1)) \cdot n \geq 0 \\
          ((p_3 - p_2) \times (q - p_2)) \cdot n \geq 0 \\
          ((p_1 - p_3) \times (q - p_3)) \cdot n \geq 0 
        \end{align*}

\end{document}
